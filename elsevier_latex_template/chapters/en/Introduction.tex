\section{Introduction}
\label{sec:Intro}

%%Example of references in Harvard style looks like this: \citet{bib:AuthorYear}.

	The automatic gain control (AGC) - is a process where output signal of device, typically electronic amplifier, is automatically kept constant by some parameter. As a parameter may be the amplitude of a simple signal or the power of the composite signal. 
	
%The amplitude of a simple signal or the power of a composite signal mau act as a controlling parameter for AGC.
	
	This process doesn't depend on the input signal.
	
	

	It was concluded that currently for transmitting signals over long distances people use these signals in frequency. As a result, high-frequency signals are formed. It means, that signals' ratio of higher and lower frequency closes to the lower one in the narrow band.
	
	% This is impossible!
	
	The developed device (module) is designed for processing the quadrature signal to the communication channel. The quadrature signal is a dimensional signal whose value at a time can be set one complex number comprising two parts, called the real part and the imaginary part. 
	
	So, developed IP-module is a part of a data path for the incoming link.

	Conventional completion of this section is the following:

	The rest of the paper is organized as follows. In section~\ref{sec:Methodology} we consider the methodology of research. In section~\ref{sec:Experiment} the dataset used is described. Section~\ref{sec:Results} contains the discussion of results obtained. Section~\ref{sec:Conclusion} concludes the paper.